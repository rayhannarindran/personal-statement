\underline{\textit{\textbf{Growing up in Indonesia,}}} the contrast between our resource-rich nation and neighboring countries 
like Singapore and Malaysia, 
who were once considered underdogs, are very striking. And I ask myself, despite our abundance, why can't we prevail in this
age of technology? And realized that the issue is not scarcity, but in our collective mindset. Many Indonesians grow up believing
success follows a single, safe path. Such as becoming doctors, businessman, or civil servants. Which leads us to be constrained
by our own expectations and fears of failure, limiting our innovation. I was raised in the same circumstances, but I choose to
make my own path, a leap of faith in my potential. This realization inspired me to pursue a more advance understanding of Computer and 
Information Engineering, as I aim to break free of those constraints and contribute to Indonesia with innovation and technological advancements.

\underline{\textit{\textbf{My academic background}}} in Computer Engineering at Institut Teknologi Sepuluh Nopember provided
me with a strong foundation in hardware and software development. Focusing on \textit{IoT} and \textit{embedded systems}, I gained
extensive hands on experience in designing and implementing intelligent solutions for real-world challenges. This was best reflected through 
my final project, where I designed, manufactured, and implemeted a \textbf{wireless cloud-based token refilling robot for prepaid electrical meters}, 
built using 3D printed components and a custom PCB. These meters are installed in nearly every household in Indonesia,
yet they can only be refilled manually and by hand, making it time consuming and inefficient. Through this project \textbf{I developed skills in
3D modelling and 3D printing, PCB design, full-stack development, and cloud deployment,} while also learning to integrate multiple disciplines into
a cohesive system. More importantly, it ignited my passion for applying engineering to create practical solutions for real use cases.

